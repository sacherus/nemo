\chapter{Experiment}


Multiple experiments were conducted using various models and different features. Let us denote some abbreviations which we will be using in this chapter:

\begin{itemize}
\item TS as the test set
\item TRS as the training set
\item WER as the word error rate
\item xEnt as the softmax loss function
\item iter. as the iteration
\item complex as the neural network with more hidden states
\item ALI\{N\} as the Nth version of alignment
\item CI as the confidence interval
\end{itemize}
 

\section{Evaluations}
If it is not written down, ``standard'' parameters of a neural network are: 2 layers, 1024 hidden neurons, 20\% cross-validation set extracted from the training test, DBN as the type of a pretraining.

\begin{tabp}[5-fold cross validation $\alpha=0.975$] 
\label{t:5_fold}
& \multicolumn{5}{c}{Fold} & \multicolumn{3}{c}{Statistics}\\
\cmidrule(r){2-6} \cmidrule(r){7-9}
 TS type &1&2&3&4&5 & $\mu$ & $\sigma$ & CI \\ 
\midrule
WER dev &24.1&23.9&23.6&24.2&23.9 & 23.94 & 0.23 & $\mu\pm 0.3$   \\ 
WER test &23.7&24.0&23.8&23.7&23.6 & 23.76 & 0.15 & $\mu\pm 0.2$ 
\end{tabp}

First experiment was conducted in order to check influence of using different subset of data (table \ref{t:5_fold}). CI for the dev and test test sets equals $\mu\pm 0.3$ and $\mu\pm 0.2$ respectively. In further experiments we will assume that if the difference is greater than CI intervals, than results are statistically significant. One should conduct experiments every time when model is changed. However, due to expensive computing time we decided to drop this procedure: it takes one day on GPU per one cell in a table.

\begin{tabp}[Reducing data set size]
\label{t:reduced1}
& \multicolumn{6}{c}{Training set size } \\
\cmidrule(r){2-7}
TS type &20\%& 35\%& 50\%& 60\% & 70\%& 80\% \\ 
\midrule
  WER dev &27.9&26.4&25.0&24.5& 24.1 & 23.9 \\ 
  WER test &29.4&27.1&25.3&24.6& 24.4 & 23.6  
\end{tabp}

\begin{tabp}[Reduced data set size. Increased complexity of the model: 6 layers, 2048 hidden neurons] 
\label{t:reduced2}
& \multicolumn{6}{c}{Training set size } \\
\cmidrule(r){2-7}
TS type &20\%& 35\%& 50\%& 60\% & 70\%& 80\% \\ 
\midrule
  WER dev &27.6&25.3& 24.2 & 23.9 &23.5&22.7 \\ 
  WER test &28.6&26.1&24.4&23.5&23.0&22.3 
\end{tabp}

In the next experiments (refer tables \ref{t:reduced1} and \ref{t:reduced2}) we ware testing how reduced data size is affecting WER results, so we would be able to use smaller training sets. As expected, increasing the training set size improve results and the biggest improvement is found when we are moving from 20\% to 35\% of the data. Because there is drop in the every step, we decided to use the maximal possible amount of the data available in the further experiments. When comparing results in the ``standard'' and ``complex'' architectures, there is 1.2\% and 1.3\% relative improvement. Although, some tests with less data (20\%) were made, when checking sanity of the scripts. Despite of including 20\% results into tables, we will not focus on them.


\begin{tabp}[Delta features on top of MFCC and better alignments] 
\label{t:better_ali}
& & \multicolumn{3}{c}{Model type} \\
\cmidrule(r){3-5}
TS type  & TRS size  &delta ALI1& delta ALI2 &delta ALI2 complex \\
\midrule
  WER dev & 20\%  &26.8& 25.7 & X \\
  WER test & 20\%  &27.7& 26.5 & X \\
  WER dev & 80\%  &X&   23.6 & 21.1 \\
  WER test & 80\%  &X& 23.0 & 20.6 
\end{tabp}

In the third experiment (table \ref{t:better_ali}) we used improved alignments (ALI2, based on GMM LDA) and delta MFCC features. We can observe improvement by 1.1\%/0.9\% with delta MFCC and by 1.1\%/1.2\% when using better alignment. We will use delta features in later and ALI2 as alignment, because WER improves significantly.

\begin{tabp}[Plain fbank and delta fbank features] 
\label{t:fbank}
&& \multicolumn{2}{c}{Feature type} \\
\cmidrule(r){3-4}
TS type & TRS size & plain & delta  \\
\midrule
  WER dev & 20\% & 25.3 & 24.2  \\
  WER test & 20\% & 26.0 & 24.5 \\
  WER dev & 80\% &  21.1 & 20.7 \\
  WER test & 80\% & 20.2 & 19.7 
\end{tabp}

As expected, fbank features improve results: 2.9\%/3.3\% when comparing with corresponding MFCC features (table \ref{t:fbank}). Using delta features instead of plain gives improvement on 0.4\%/0.5\% level.

\begin{tabp}[LDA + MLTT feature type on the top of MFCC] 
\label{t:lda}
& \multicolumn{4}{c}{Feature type} \\
\cmidrule(r){2-5}
 & LDA GMM & LDA & LDA delta & delta LDA \\
\midrule
TRS size & 100 & 20 & 20 & 80 \\
WER dev & 26.6 & 26.2  & 25.3 & 21.9\\ 
WER test & 25.4 & 25.4  & 24.9 & 20.4
\end{tabp}

In the table \ref{t:lda} we tried using decorrelated features (LDA), what is not advised in using a neural network, because it can decorrelate features using multiple hidden layers. We decided to stick to the fbank features. However, LDA results brought surprise.

\begin{tabp}[Improved LM - dictionary with probabilities in the lexicon reflecting state of the training set. 0.80 training set size was used]
	\label{t:dictionary}
& \multicolumn{2}{c}{Feature type} \\
\cmidrule(r){2-3}
TS type & delta LDA & delta fbank \\
\midrule
  WER dev &  21.7 & 20.7 \\ 
  WER test &  20.0 & 19.4
\end{tabp}

When reading original scripts, we discovered that we can improve the LM by improving the lexicon: adding probabilities of the variants of the words from training set. We found the results being almost in the previously defined confidence interval: -0.2\%/0.4\% for LDA and -0.0\%/0.2\% for fbank (table \ref{t:dictionary}).

\begin{tabp}[Comparison of DNNs with and without pretraining]
	\label{t:pretraining}
& \multicolumn{2}{c}{6 hidden layers} & \multicolumn{3}{c}{2 hidden layers}\\
\cmidrule(r){2-3} \cmidrule(r){4-6}
Metric/Initialization &  random & RBM & random & RBM & random \\
Set size &  80 & 80 & 80 & 80 & 20 \\
\midrule
WER dev &  18.2 & 18.1 & 20.7 & 20.7 & 25.0  \\ 
WER test &  17.5 & 17.0 & 19.8 & 19.4 & 25.5 \\
xEnt 1st iter. & 2.89/3.32 & 2.35/2.71 &2.8/3.36 &2.46/2.91 & \\
xEnt last iter.  & 2.16/2.52 & 1.9/2.1 & 2.39/2.65  & 2.03/2.28 & \\
FA 1st iter. & 35.18/28.19 & 44.45/35.69 & 37.1/28.0 & 42.97/34.54 & \\
FA last iter. & 45.52/39.46 & 53.49/47.5 & 42.1/37.6 & 49.26/44.57 & 
\end{tabp}

As the most important aspect of this work we are comparing 2 types of the pretraining of the neural network: RBM and random initialization (table \ref{t:pretraining}. Unlike to our previous experiments, we weren't able to show that RBM pretraining method boosted our final metric WER. However, we noticed that:

\begin{itemize}
	\item Softmax function lost after first iteration is much lower when using pretraining. 
	\item After last iteration there is still improvement in xEnt, but not as big as after first iteration.
	\item It is also worth to notice very high frame accuracy on models with pretraining.
\end{itemize}

\begin{comment}
\begin{tabp}[LSTM network]
	\label{t:lstm}
Feature type & LSTM & LSTM \\ 
Set size & 20 & 80 \\
\midrule
WER dev & 22.1 & 18.4 \\
WER test & 21.0 & 17.4
\end{tabp}
\end{comment}

\section{Conclusions}
Let us quote our theses: 
``\textbf{In this work we will adopt DL~to Automatic speech recognition (ASR) in this work and show its advantageous over most common technique which is the Gaussian mixture models (GMM).}'' and ``\textbf{Our goal is to use of one of a deep learning techniques and show its superiority over standard approaches like GMMs and shallow architecture. The confirmation or the rejection of this thesis can be found in the part ``Experiments, data, results''.}''. Now we are claiming that our theses are confirmed.

Our first conducted experiments showed that there is small variety in the 5-fold validation, so we assumed that experiments which differ with the only $\pm 0.3$ WER are not statistically significant. Therefore, we set up framework which allow us to conduct experiments and check if any method is improving results. On the other hand we did not reduce data size, because adding 10\% more data is decreasing WER by 1\% when moving from 70\% to 80\%. With such posterior knowledge we started moving toward the final experiment.

Before conducting the final experiment we started examine various feature types. What we found agree with our expectations: ``Fbank features are superior over MFCC'' \parencite{hinton_deep_2012}. Using more complex RBMs, better alignment on top of the better features (LDA) helps improving WER. \textbf{What is interesting, training neural networks using decorrelated features helps almost good as the using Fbank features. Here there is place for further improvement: training neural networks using the LDA, the Fbank features and the neural networks.}

\textbf{Finally we conducted experiment in which we used one of the deep learning algorithms: deep belief networks and we reported similar results to work done by: \textcite{mohamed_phone_2010}}. However, WER metric did not improved significantly. 

\section{Further work}
Despite of successfully adopting one of the deep learning techniques we were not able to improve results in a sense of the WER. Therefore our proposal for further experiments is:
\begin{enumerate}
\item investigate why there is no significant improvement in sense of the WER
\item surveying all deep learning techniques
\item using LDA on top of the Fbank features
\item using pure frames as the features (only with windowing or event without it)
\end{enumerate}

\chapter{Digital Signal Processing}

Signal $x_a(t), -\infty < t < \infty$ is a~function of independent variable which is the time. In order to digitize signal and be able to store it into memory we must digitize signal and create sequence or a~function. The process is done using Analog Digital converter - device which converts analog to digital signal using a~sampler and a~quantizer.

The sampler ``takes samples'' of the analog signal $x_a(t)$ every $T_s$ seconds. The sampling frequency $F_s$, the sampling period, independent variable - time($t$) and independent variable - sample (n) are related by formula:

\begin{equation}
	t = nT_s = \frac{n}{F_s}
\end{equation}

\begin{equation}
	x_a(nT) \equiv x(n)
\end{equation}


Most often functions used in signal analysis are a~cosine and sine waves in the form:
\begin{equation}
	x_a(t) = Acos(\Omega t +~\theta)
\end{equation}

\begin{equation}
	\Omega \equiv 2\pi F
\end{equation}

Where A~is an amplitude signal strength, $\Omega$ - frequency (rad/s), F - Hz, $\theta$ - phase (rad) shift.  
Similar we can define those variables in digital domain:

\begin{equation}
	x(n) = Acos(\omega t +~\theta)
\end{equation}

\begin{equation}
	\omega \equiv 2\pi f
\end{equation}


Where A~is an amplitude signal strength, $\omega$ - frequency (radians/sample), f - (cycles/sample), $\theta$ - phase (rad) shift.  

N~is a~period of the~signal when


\begin{align}
& x(n+N) = x(n), \forall n \in \mathbb{Z} \\
& cos(2\pi f (n +~N) +~\theta) = cos(2\pi f n +~\theta) \\
& f = \frac{k}{N}, k \in \mathbb{Z} \implies f \in \mathbb{Q}
\end{align}

This implies that frequency in digital time domain is a rational number.


Equation:
\begin{equation}
	x_k(n)=Acos(2\pi f_0(n) +~\theta),
\label{cosine_digital}
\end{equation}
is periodic, where period of $x_k$ equals


\begin{equation}
	2 \pi f = 2 \pi f_o +~2\pi k \implies f = f_0 + k, k \in \mathbb{N}
\end{equation}
This allow to distinguish range for fundamental frequency $f_0$
\begin{equation}
|f_0| < \frac{1}{2}
\label{fundamental_one_half}
\end{equation}
Frequencies $|f| > \frac{1}{2}$ are called ``alias frequencies'' which ``mirror'' the~fundamental frequency from range $f_0$.


When we sample an analog cosine signal $x_a$ with the sampling frequency $F_s$, we get:

\begin{equation}
x_a(nT)=x_a(\frac{n}{F_s}) \equiv x(n) = Acos(\frac{2\pi n F}{F_s} +~\theta) 
\label{eq:cosine_sampling}
\end{equation}

Comparison of equations \eqref{eq:cosine_sampling} and \eqref{cosine_digital} yields 

\begin{equation}
f_0 = \frac{F}{Fs}
\label{fundamental_sampling}
\end{equation}

Substituting $f_0$ in \eqref{fundamental_one_half} by \eqref{fundamental_sampling} results in one of the most important conclusions in digital signal processing which is maximum frequency one can digitize:

\begin{equation}
	\abs*{\frac{F}{F_s}} < \frac{1}{2} \implies \abs{F} < \frac{F_s}{2}
\end{equation}
 

Maximum frequency which could be converted to digital domain is half of the sampling frequency - is called ``folding frequency'' $F_{fold}$. On the other hand having maximum frequency $F_{max}$ in our signal $x_a$ one is able to define ``Nyquist rate'', which is equal to $2 F_{max}$. Human ear is working in range $20 Hz <~F_h <~20kHz$, which yields ``Nyquist rate'' - the sampling frequency - 40kHz (often used frequency 44.1kHz due to imperfection of low\dywiz band filters). Speech lies in the range $< 16kHz$. Other frequencies $> 16kHz$ should be filtered\dywiz out by analog low\dywiz band filters. This subject is out of the scope of this work. 

An input signal is analog data which contains continuous range of values, therefore it should be quantize to fit into memory range. Quantization error should be introduced, because of quantizer ($Q[x(n)] \equiv x_q(n)$):

\begin{equation}
e_q(n) = x_q(n) - x(n)
\end{equation}

The problem can be solved by rounding or by truncation. In this work we will focus only on rounding. Error $e_q(n)$ is limited to:


\begin{equation}
	-\frac{\Delta}{2} < e_q(n) < \frac{\Delta}{2}
\end{equation}
where $\Delta$ means:

\begin{equation}
	\Delta = \frac{x_{max}-x_{min}}{L-1}
\end{equation}
where $x_{max}, x_{min}$ are the maximum and minimum values of the signal, L~is the number of quantization levels.


\begin{align}
	& x(n) = 1 \\
	& x(n) = 1 
\end{align}

\todo{This like for proletariat}

\missingfigure{Example sentence in audacity}



\chapter{Introduction}
In the last 5 years ``Machine Learning'', ``Big Data'', ``Neural Nets'', etc. become buzz words and hot topics in science and industry. Business often doesn't understand them. Nevertheless, a~great amount of money is put in the area of ``Big Data''. Development of the new algorithms and techniques is highly is desirable, because some problems (NP problems) are too complex for programmers to calculate all possible combinations (like k\dywiz means clustering published by Lloyd in 1957) or simple direct method does not exist (part\dywiz of\dywiz speech tagging).  

Machine learning (ML) scientific term was coined by \textcite{samuel} as ``Programming computers to learn from experience should eventually eliminate the need for much of this detailed programming effort''. Deep learning (DL) one of the most prominent and recent ML techniques has brought the improvement of results and it is widely used by the industry. In this work we will try show that DL~ overwhelmed other ML algorithms in some real life application.

\textbf{In this work we will adopt DL~to Automatic speech recognition (ASR) in this work and show its advantageous over most common technique which is the Gaussian mixture models (GMM).}

This work is divided into three parts:

\textbf{Automatic speech recognition} defines basic concepts and is briefly describes history of a speech recognition. One can find the idea which will be proofed later. Also it contains a summary of approaches which has been used so far but on the smaller scale.

\textbf{Models and features extraction} which is the most voluminous part of this work and derivative algorithms and definitions that will be used later in our experiments.

\textbf{Experiment, data and results} that gives origin of the data, descriptions of all all metrics used in experiments, summary of our experiments and final discussion which will allow to confirm or reject our thesis. 
	


\chapter{Introduction}
In the last 5 years ``Machine Learning'', ``Big Data'', ``Neural Nets'', etc. become buzz words and hot topics in science and industry. Business often doesn't understand them. Nevertheless, a~great amount of money is put in the area of ``Big Data''. Development of the new algorithms and techniques is highly is desirable, because some problems (NP problems) are too complex for programmers to calculate all possible combinations (like k\dywiz means clustering published by Lloyd in 1957) or simple direct method does not exist (part\dywiz of\dywiz speech tagging).  

Machine Learning (ML) scientific term was coined by \textcite{samuel} as ``Programming computers to learn from experience should eventually eliminate the need for much of this detailed programming effort''. Deep Learning (DL) one of the most prominent and recent ML techniques has brought the improvement of results and it is widely used by the industry. In this work we will try show that DL~ overwhelmed other ML algorithms in some real life application.

Variety of everyday life problems can be helped by Machine Learning methods. We will adopt DL~ to Automatic speech recognition (ASR) in this work and show its advantage over most common technique which is gaussian mixture models (GMM).

This work contains:
